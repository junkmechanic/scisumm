The first step was to build a basic system that could be the basline for further experiments and improvements.
Another factor for consideration was that the aim of the project was to create an extractive summary consisting of sentences from the same document that needs to be summarized.
After studying various approaches used by researchers for single document summarization, I decided to use an unsupervised ranking algorithm.
Following is a brief description of the TextRank algorithm, as introduced by \emph{Mihalcea et. al.[]}, to explain how a graph based raking algorithm can be applied to rank sentences within a document.

\subsection*{TextRank}
The basic idea implemented by a graph-based ranking model is that of 'voting' or 'recommendation'.
When one vertex links to another one, it is basically casting a vote for the other vertex.
The importance of a vertex is based on the number of votes casted towards that vertex. 

Formally, let \emph{G = (V, E)} be a directed graph with the set of vertices \emph{V} and set of edges \emph{E} where is a subset of \emph{{V x V}}.
For a given vertex \(V_i\), let \(In(V_i)\) be the set of vertices that point to it, and let \(Out(V_i)\) be the set of vertices that \(V_i\) points to.
The score of vertex \(V_i\) is defined as follows (\emph{Brin and Page []}),
\[S(V_i) = (1 - d) + d * \sum_{j \in In(V_i)} \frac{1}{|Out(V_j)|}S(V_j)\]
where \emph{d} is the damping factor that can be set between 0 and 1.
The factor \emph{d} is generally set to \emph{0.85} and this the value that was used for my experiments as well.
Starting with arbitrary values assigned to each node in the graph, the computation iterates untill convergence below a given threshold is achieved.
The final score associated with each vertex represents the \textit{importance} of that vertex within the graph.

Two modifications were introduced by \emph{Mihalcea []} to apply this algorithm to documents for ranking sentences.
The first was to define the above mentioned algorithm for undirected graphs in which case, the out-degree of the vertex is equal to the in-degree of the vertex.
The other modification was to incorporate the notion of \textit{strength} of the connection between any two vertices as a weight \(w_ij\) added to the corresponding edge that connects the two vertices.
Consequently, the new formula that takes into account these two modifications and gives the weighted score is
\[WS(V_i) = (1 - d) + d * \sum_{V_j \in In(V_i)} \frac{w_ji}{\sum_{V_k \in Out(V_j)}w_jk}WS(V_j)\]

\subsection*{TextRank for Summarization}
