\section{Problem Definition}
A summary is a text that represents and preserves the important informaiton conveyed by a document in a concise manner.
Automatic summarization has been an area of research for the last couple of decades and has been tackled with different approaches.
For my Final Year Project, I have decided to address the problem of summarization of scientific documents and put forward a solution in the form of a pipeline to process scientific articles and output their summaries.

In order to summarize a document, it is first necessary to identify what pieces of text within the document are important enough to represent the information conveyed by the document.
Such pieces of text can range from single words (motivated by the problem of keyword extraction) to clauses and sentences within the document to be summarized.
For this project, I will focus on complete sentences for the reason that a sentence represents a complete coherent line of thought compared to using clauses.
So it would be easier to plug the chosen sentences into the summary without worrying about any significant loss of continuity.

Another point to be considered is that summaries are generaly meant to be concise and limited in the number of words.
It follows that not all information, that has been identified as important, might fit into a summary.
Hence, there is a need to rank and choose the most important pieces of informaiton (sentneces) to be included into the summary.

Eventually, all the high ranked sentences are clubbed together and presented as a summary in what is termed as \textit{extractive summarization}.
A further improvement over creating extracts for summaries is to create an abstract of the assembled text which is a more recent and more difficult problem.
However, in this study, I focus on creating extractive summaries.

To limit the domain of analysis, I dediced to use scientific articles from the ACL Anthology.

\section{Challenges with Summarization}
Several summarization algorithms have been suggested in the past for creating extractive summaries.
Some are simple algorithms like picking the first few lines of the article to be summarized or picking random sentences from the article.
On the other hand, there are other algorithms that rank sentences based on the informative value of the content in the sentences.

The performance of such algorithms is judged based on the ROUGE score [C-Y Lin \cite{Lin2004}].
ROUGE score for a summarization algorithm is calculated by comparing the summaries generated by this algorithm against gold standard summaries (human summaries) for the corresponding articles.
However, the interesting point is that each summarization algorithm has its advantages depending on the domain and the type of problem.
For example, for summarization of news articles, directly picking the first few lines of the text has proven to be a strong baseline [Nenkova \cite{DBLP:conf/aaai/Nenkova05}].
Whereas for summarization of short articles, a rank based algorithm has been shown to perform much better than just picking the first few lines.

Considering that the corpus used for this project consists of scientific articles from the ACL Anthology, these articles are generally about a specific study and elaborate on the methodology of the conducted experiments, the results obtained, conlusion and possible discussion arising from conducting the experiments.
These also include an "Introduction" section as well as a "Related Work" section to describe the work of others who have addressed the same or similar problem.

Prevalent summarization algorithms do not discriminate sentences on the basis of discourse.
Hence their performance on scientific articles is expected to be low.
This has been confirmed in this study as will be described in later sections.

\section{Group Affiliation}
I performed this study under the supervision of Asoc. Prof. Min Yen Kan at the School of Computing, National University of Singapore.
I am a member of a group of 5 individuals who are being advised by Prof. Kan and are working towards creating a Scientific Document Summarization System.
The problem is inspired by the upcoming shared task which would involve a similar problem.
My group is trying to tackle this problem from two perspectives.
One is to look at the source document content to generate a summary.
The other is to consider other scientific articles that cite the source article and analyse these citation sentences.
This could be beneficial because such citation sentences generally describe the work done in the source article in a couple of lines.

I am responsible for preparing the system that analyses the source document content to generate an extractive summary.