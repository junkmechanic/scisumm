A summary is a text that represents and preserves the important informaiton conveyed by a document in a concise manner.
Automatic summarization has been an area of research for the last couple of decades and has been tackled with different approaches.
In this paper we will address the problem of summarization of scientific documents and put forward a solution in the form of a pipeline to process scientific articles and output their summaries.

In order to summarize a document, it is first necessary to identify what pieces of text within the document are important enough to represent the information conveyed by the document.
Such pieces of text can range from single words (motivated by the problem of keyword extraction) to clauses and sentences within the document to be summarized.
In this paper, we will focus on complete sentences for the reason that a sentence represents a complete coherent line of thought compared to using clauses.
So it would be easier to plug the chosen sentences into the summary without worrying about any significant loss of continuity.

Another point to be considered is that summaries are generaly meant to be concise and limited in the number of words.
It follows that not all information, that has been identified as important, might fit into a summary.
Hence, there is a need to rank and choose the most important pieces of informaiton (sentneces) to be included into the summary.

Eventually, all the high ranked sentences can be clubbed together and presented as a summary in what is termed as \textit{extractive summarization}.
A further improvement over creating extracts for summaries is to create an abstract of the assembled text which falls in the domain of language generation.

In this study, we will focus on creating extractive summaries. We started our study by building a baseline to rank sentences based on the relevance of words that occur in each sentence.
The measure of relevance depends intuitively on how often words appear in different sentences in the document under consideration.
Based on this overlap of words among different sentences, the ranking algorithm gives a high score to a sentence if it has a large number of overlaps with other sentences in the same document.
The details of this algorithm , TextRank, proposed by \citep{•}, are discussed in Section 3.
To improve the information content of the summary, 

********Describe issues and challenges********