%
% File acl2014.tex
%
% Contact: koller@ling.uni-potsdam.de, yusuke@nii.ac.jp
%%
%% Based on the style files for ACL-2013, which were, in turn,
%% Based on the style files for ACL-2012, which were, in turn,
%% based on the style files for ACL-2011, which were, in turn, 
%% based on the style files for ACL-2010, which were, in turn, 
%% based on the style files for ACL-IJCNLP-2009, which were, in turn,
%% based on the style files for EACL-2009 and IJCNLP-2008...

%% Based on the style files for EACL 2006 by 
%%e.agirre@ehu.es or Sergi.Balari@uab.es
%% and that of ACL 08 by Joakim Nivre and Noah Smith

\documentclass[11pt]{report}
\usepackage{times}
\usepackage{url}
\usepackage{latexsym}
%% Pictures
\usepackage{graphicx}
%% For multirow in tables
\usepackage{multirow}
%% Algorithm
\usepackage{algorithm}
\usepackage{algorithmicx}
\usepackage{algpseudocode}
%% For math notations
\usepackage{amsmath}
\usepackage{amsfonts}
%% Text coloring
\usepackage{color}
%% To control enum spacing for brevity
\usepackage{enumitem}

%\setlength\titlebox{5cm}

\definecolor{light-gray}{gray}{0.95}


% Fix differences in cite commands
\newcommand{\citep}[1]{\cite{#1}}
\newcommand{\citet}[1]{\newcite{#1}}

% You can expand the titlebox if you need extra space
% to show all the authors. Please do not make the titlebox
% smaller than 5cm (the original size); we will check this
% in the camera-ready version and ask you to change it back.
\newcommand{\EE}{\emph{E-E}}
\newcommand{\ET}{\emph{E-T}}
%\newcommand{\SWING}{\texttt{SWING}}
%\newcommand{\TASWING}{\texttt{TA-SWING}}
%\newcommand{\CLASSY}{\texttt{CLASSY}}
%\newcommand{\POLYCOM}{\texttt{POLYCOM}}
\newcommand{\R}{$\mathbb{R}$}
\newcommand{\LL}{$\mathbb{L}$}




\title{Automatic Summarization of Scientific Articles}

%
%\author{Ankur Khanna \\
%  National University of Singapore\\
%  Affiliation / Address line 2 \\
%  Affiliation / Address line 3 \\
%  {\tt ankurkhanna@nus.edu.sg} \\\And
%  Min-Yen Kan \\
%  National University of Singapore\\
%  Affiliation / Address line 2 \\
%  Affiliation / Address line 3 \\
%  {\tt kanmy@comp.nus.edu.sg} \\}

\date{}

\begin{document}
\maketitle
\begin{abstract}
We propose a pipeline for summarizing scientific articles.
The pipeline consists of a baseline algorithm, TextRank, to rank sentences within a document based on relevance of the words within each sentence.
To improve the selection of important sentences, each ranked sentence is then given as input to an SVM classifier to decide whether the sentence should be included in the summary or not.
The ROUGE results show an increase in the scores over the baseline and we also present an analysis of the why our algorithm performs better than the baseline.
\end{abstract}

\section{Introduction}
\label{section:introduction}
A summary is a text that represents and preserves the important informaiton conveyed by a document in a concise manner.
Automatic summarization has been an area of research for the last couple of decades and has been tackled with different approaches.
In this paper we will address the problem of summarization of scientific documents and put forward a solution in the form of a pipeline to process scientific articles and output their summaries.

In order to summarize a document, it is first necessary to identify what pieces of text within the document are important enough to represent the information conveyed by the document.
Such pieces of text can range from single words (motivated by the problem of keyword extraction) to clauses and sentences within the document to be summarized.
In this paper, we will focus on complete sentences for the reason that a sentence represents a complete coherent line of thought compared to using clauses.
So it would be easier to plug the chosen sentences into the summary without worrying about any significant loss of continuity.

Another point to be considered is that summaries are generaly meant to be concise and limited in the number of words.
It follows that not all information, that has been identified as important, might fit into a summary.
Hence, there is a need to rank and choose the most important pieces of informaiton (sentneces) to be included into the summary.

Eventually, all the high ranked sentences can be clubbed together and presented as a summary in what is termed as \textit{extractive summarization}.
A further improvement over creating extracts for summaries is to create an abstract of the assembled text which falls in the domain of language generation.

In this study, we will focus on creating extractive summaries. We started our study by building a baseline to rank sentences based on the relevance of words that occur in each sentence.
The measure of relevance depends intuitively on how often words appear in different sentences in the document under consideration.
Based on this overlap of words among different sentences, the ranking algorithm gives a high score to a sentence if it has a large number of overlaps with other sentences in the same document.
The details of this algorithm , TextRank, proposed by \citep{•}, are discussed in Section 3.
To improve the information content of the summary, 

********Describe issues and challenges********

\section{Related Work}
\label{section:related work}
One of the earliest work done in summarization is that of Luhn \cite{luhn} in which he used the frequecy of words after stemming and removing stopwords.
A significant factor for a sentence was derived from the number of occurances of significant words (words with high frequency) and the distance between significant words in that sentence.
The top ranked sentences based on this significance factor were selected to form the summary extract.
Later features like sentence position (Baxendale \cite{Baxendale}) and cue words (Edmundson \cite{Edmundson:1969:NMA:321510.321519}) were experimented with as well.

With the advent of NLP, researchers produced started to produce works with statistical learning.
For example Aone et. al. \cite{Aone99} used term frequency and inverse document frequency as well as shallow discourse analysis (like reference to same entities in the text) to train a naive-Bayes classifier.
Conroy and O'Leary \cite{Conroy:2001:TSV:383952.384042} modeled the problem of extracting a sentence from a document using a hidden Markov model (HMM).
The basic motivation for using a sequential model is to account for local dependencies between sentences.
Only three features were used: position of the sentence in the document (built into the state structure of the HMM), number of terms in the sentence, and likeliness of the sentence terms given the document terms.

Svore et. al. \cite{Svore} propose an algorithm based on neural nets and the use of third party datasets to tackle the problem of extractive summarization, outperforming the baseline with statistical significance.
They trained a model from the labels and the features for each sentence of an article, that could infer the proper ranking of sentences in a test document.

The work mentioned above involves training on corpus and hence is supervised.
Unsupervised algorithms have also been proposed for this task.
Mihalcea el al. \cite{mihalcea-tarau:2004:EMNLP} have proposed an algorithm, TextRank which is similar to PageRank \cite{ilprints422}.
TextRank is a graph based ranking algorithm which can be used to rank keyword phrases or sentences based on the content overlap with other sentences in the document.
This has been further discussed later in the report.

Xie et al. \cite{4518777} have discussed different metrics that could be used to calculate the importance of sentence content.
They use such metrics to find the best set of sentences that could represent the entire document based on the content.
The ranking algorithm used in this work calculates the Maximum Marginal Relevance \cite{Carbonell:1998:UMD:290941.291025} of sentences, which can be used as a score to rank sentences.

Lin et al. \cite{conf/asru/LinBX09} have also proposed a graph based algorithm where the document is represented as a set of sentences (V) in a graph with the edges defining the similarity between sentences.
They define submodular functions using these similarity values to map a set of extracted sentences to a value that can quantify the efficiency of the summary.
They use the greedy algorithm for this discrete optimization problem to incrementally add elements to the set that maximize the output.
They report that this method outperforms MMR and TextRank methods.
The summaries were created for meetings.


\section{Methodology}
\label{section: method}
The first step was to build a basic system that could be the basline for further experiments and improvements.
Another factor for consideration was that the aim of the project was to create an extractive summary consisting of sentences from the same document that needs to be summarized.
After studying various approaches used by researchers for single document summarization, I decided to use an unsupervised ranking algorithm.
Following is a brief description of the TextRank algorithm, as introduced by \emph{Mihalcea et. al.[]}, to explain how a graph based raking algorithm can be applied to rank sentences within a document.

\subsection*{TextRank}
The basic idea implemented by a graph-based ranking model is that of 'voting' or 'recommendation'.
When one vertex links to another one, it is basically casting a vote for the other vertex.
The importance of a vertex is based on the number of votes casted towards that vertex. 

Formally, let \emph{G = (V, E)} be a directed graph with the set of vertices \emph{V} and set of edges \emph{E} where is a subset of \emph{{V x V}}.
For a given vertex \(V_i\), let \(In(V_i)\) be the set of vertices that point to it, and let \(Out(V_i)\) be the set of vertices that \(V_i\) points to.
The score of vertex \(V_i\) is defined as follows (\emph{Brin and Page []}),
\[S(V_i) = (1 - d) + d * \sum_{j \in In(V_i)} \frac{1}{|Out(V_j)|}S(V_j)\]
where \emph{d} is the damping factor that can be set between 0 and 1.
The factor \emph{d} is generally set to \emph{0.85} and this the value that was used for my experiments as well.
Starting with arbitrary values assigned to each node in the graph, the computation iterates untill convergence below a given threshold is achieved.
The final score associated with each vertex represents the \textit{importance} of that vertex within the graph.

Two modifications were introduced by \emph{Mihalcea []} to apply this algorithm to documents for ranking sentences.
The first was to define the above mentioned algorithm for undirected graphs in which case, the out-degree of the vertex is equal to the in-degree of the vertex.
The other modification was to incorporate the notion of \textit{strength} of the connection between any two vertices as a weight \(w_ij\) added to the corresponding edge that connects the two vertices.
Consequently, the new formula that takes into account these two modifications and gives the weighted score is
\[WS(V_i) = (1 - d) + d * \sum_{V_j \in In(V_i)} \frac{w_ji}{\sum_{V_k \in Out(V_j)}w_jk}WS(V_j)\]

\subsection*{TextRank for Summarization}


\section{Experiments and Results}
\label{section:experiments}
\input{600_experiments.tex}

\section{Discussion}
\label{section:discussion}
From the results in the previous section, it can be seen that the performance of the Classifier module over the test set was better than the performance of the baseline.
It shows that considering such features that rely on computation of relevance of the content within a sentence can give much better results than considering just the textual content of the sentence.
This can be attributed to the fact that in the Classification module, the features extracted for classification of the sentence captured the relevance of the the words (that appear in that sentence) with respect to the entire document.

Although the test set consisted of sentences that were all selected from the top of the output of the TextRank algorithm, the negative sentences in the training set were picked from the bottom of the output of the TextRank algorithm.
From the confusion matrix in Table 4.2 we can see that the negative predictive value (which indicates how well negative samples are classified) of the trained model is 62.2\% and the specificity (which indicates the fraction of true negative samples that were identified correctly) is 93.1\%.
This shows that the trained model is able to identify the negative sentences in most of the cases.
This helps in validating the decision to pick negative samples as the the lowest ranked sentences in the document for training.

Despite the accuracy of the trained model in predicting negative sentences, it is still more important to judge the classifier based on its efficiency to classify positive sentences correctly.
This is because the aim of the task is to highlight sentences that could be important, which are positive sentences in the data set.
A close analysis of the sentences in the test set, their respective feature values and the real and predicted classes for each helped in understanding the shortcomings of the model.
Since the importance metric sec-tf-idf is calculated by considering the section (as annotated in the document structure) as a document and the whole document as the corpus, the value of the same word differs with its occurance in different sections.

Figure 5.1 shows the dependency parse of a positive sentence which was classified correctly by the trained model.
The values of the three features have also been mentioned beside the words that represent the subject, the verb and the object phrases.
Although it is an estimate in accordance with the plots shown in Figure 4.1, it can be seen that these values are high enough to be considered as positive samples.

\begin{figure}[h]
\begin{subfigure}{0.5\textwidth}
\includegraphics[width=1.0\linewidth, height=7cm]{discuss1} 
\caption{A true positive sentence}
\label{fig:dep-discus1}
\end{subfigure}
\begin{subfigure}{0.5\textwidth}
\includegraphics[width=1.0\linewidth, height=7cm]{discuss1}
\caption{A false positive sentence}
\label{fig:dep-discus2}
\end{subfigure}
\label{fig:stanford-dep-parse}
\end{figure}

Figure 5.2 shows the dependency parse of another positive sample which was classified incorrectly by the trained model as a negative sentence.
Here it can be seen that the value of the word 'propose' is 0.007 which is quite low where as the value of the same word in Figure 7.1 is 0.08.
The values of this word are different because the two sentences appear in different sections.

I used the value of the word 'propose' from the first sentence as a feature for the second sentence since in both the sentences this word is the main verb and is used for that feature.
With this and the other two values of subject and object phrases from the second sentence, I created a feature vector and used the trained model to classify it.
The feature vector was classified as a positive sample.

From the above discussion it can be seen that although the feature sec-tf-idf can be used to perform classification that could lead to imporvement over the baseline, it can still miss out on positive sentences.
Although the precision of the train model using this novel metric is sufficiently good, the recall is poor.
For summarization, it is important to find sentences that could be included in the summary.
Hence, there is a need to improve the features to help better map the importance of words in a sentence.

A further improvement that has been proposed for future work is to add a linear component to the features being calculated.
a feature that measure a more general significance of words can be used.
This can be measure over large corpus which need not necessaily be a collection of scientific articles.
A possible suggestion is the use of Google n-gram data set.
The feature could be a linear combination of the sec-tf-idf value as well as Google n-gram data set.

\section{Conclusion}
\label{section:conclusion}
In this report, I documented the research conducted by me along with my group to develop a pipeline for summarizing scientific articles.
The baseline was built using TextRank algorithm which is unsupervised and hence produces a output in the form of ranked sentences based on content overlap between sentences.
This output can be used for futher processing.

Experiments were conducted to build a classification model that can identify the importance of sentences within a document.
Different metrics were tried as features for the words and phrases representing the verb, subject and object of a sentence.
A novel metric was proposed which measure the importance of words with respect to the seciton that the sentence appears in.
The classification model build using this feature set was seen to perform better than the baseline.
The final pipeline was used to create summaries which were judged against gold standard and it performed better than the baseline.

An in depth analysis of this approach helped in understanding the strength and weakness of the feature set and the values.
This has helped in planning for the next phase in terms of deicding the improvements to be done to the features in the next iteration of experiments.


%%%%%%%%%%%%%%%%%%%%%%%%%%
%% Taken out for initial submission
%%\section*{Acknowledgments}
%%
%%The acknowledgments should go immediately before the references.  Do
%%not number the acknowledgments section. Do not include this section
%%when submitting your paper for review.
%%%%%%%%%%%%%%%%%%%%%%%%%%

%% Bibliography
\bibliographystyle{acl}
\bibliography{900_bibliography}


\end{document}
